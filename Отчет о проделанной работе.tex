\documentclass[12pt,a4paper]{scrartcl}
\usepackage[utf8]{inputenc}
\usepackage[english,russian]{babel}
\usepackage{indentfirst}
\usepackage{misccorr}
\usepackage{graphicx}
\DeclareGraphicsExtensions{.pdf,.png,.jpg}
\graphicspath{{/}}
\usepackage{amsmath}

\begin{document}

\begin{center}
\begin{large}
Проект Кока-Кола
\end{large}
	\bigskip
     
Ахундзянов Амир Андреевич\\
Шуббе Леонтий Павлович
\end{center}

\section{Задачи и цели}
В ходе данного проекта планируется придумать методы для высокоточного сбора информации о состоянии жидкости  содержащей угольную кислоту и газа, находящихся и изолированном сосуде. После получения такой установки можно привести первичный анализ полученных данных и понять что-нибудь интересное про процесс выхода большого количества газа при встряске.

\section{Теоретическое обоснование}



\section{Методика и процесс создания оборудования}
Первоначально планировалось использовать механический манометр со стрелкой для измерения давления внутри бутылки с кока-колой и по диапазону измерений нам идеально подходил автомобильный манометр, но его использование вело к усложнению процесса сбора данных, так как значения приходилось снимать вручную, а изучемый процесс выхода газов мог идти около двух суток. Также точность измерения такого манометра далека от идеала. По этим и другим причинам было решено использовать электронный датчик давления MPXHZ6400AC6T1 , подключенный к микроконтроллеру Arduino Uno. Это позволило снимать показания без непосредственного участия человека и снизить погрешность до порядка 500 Паскалей. Таким образом, вклеив датчик в стандартную крышку для пластиковых бутылок, спаяв схему и написав прошивку для микроконтроллера мы собрали установку, измеряющую зависимость давления от времени.

Далее хотелось проводить такое исследование при постоянных, но разных температурах. Для охлаждения системы требуется сложное специализированное оборудование, поэтому мы удовлетворились одним измерением в холодильнике. Повышать же температуру много проще. Достаточно использовать простой нагреватель, как, наример, кипятильник. Поэтому следующим дополнением установки было подключение к Arduino реле, замыкающего питание нагревателя, опущенного в кастрюлю с водой, в которой лежала исследуемая бутылка. Для поддержания постоянства температуры воды вне занисимости от внешних условий был использован электронный датчик температуры DS18B20 в герметичном корпусе. При использовании наивного алгоритма поддержания температуры постоянной, который включает нагрев при температуре ниже заданной и отключает при температуре выше, из-за задержке в распространения тепла по кастрюле, металлический нагреватель успевал сильно нагреться и после отключения питания успевал сильно нагреть воду.

\begin{LARGE}
\textbf{\textit{Добавить картинку с графиком}}
\end{LARGE}

Поэтому прошивка предполагала включение нигрева на заданное время и последующее ожидание выравнивания температуры при большем понижении температуры время этого ожидания уменьшалось. Параллельно со всем этим проводилась 
\section{Результаты измерений и обработка данных}



\section{Вывод}




\end{document}
